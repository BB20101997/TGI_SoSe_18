\section*{Aufgabe 3}

\subsection*{a)}
	Sei M deterministische TM die $CIRCLE^5$ folgendermaßen löst:
	
	1) Für $e_1$ Kante in $G$
	
	2) Für $e_2$ Kante in $G$

	3) Für $e_3$ Kante in $G$

	4) Für $e_4$ Kante in $G$

	5) Für $e_5$ Kante in $G$

	6) Überprüfe of die Kanten $e_1$ bis $e_5$ einen Kreis Bilden.
	
	7) Wenn $e_1$ bis $e_5$  einen Kreis bilden akzeptiere.

	8) Wenn $e_5$ nicht letzte Kante in $G$ wähle nächste Kante bei 5)

	9) Wenn $e_4$ nicht letzte Kante in $G$ wähle nächste Kante bei 4)

	10) Wenn $e_3$ nicht letzte Kante in $G$ wähle nächste Kante bei 3)

	11) Wenn $e_2$ nicht letzte Kante in $G$ wähle nächste Kante bei 2)

	12) Wenn $e_1$ nicht letzte Kante in $G$ wähle nächste Kante bei 1)
	
	13) verwerfe
	
	Dann ist M's Laufzeit beschränkt durch $E^5$ mit $E$ anzahl der Kanten.
	
\subsection*{b)}
	
	Sei M eine TM die $CIRCLE^{half}$ folgendermaßen löst:
	
	1) Setze Zähler $n = 0$
	
	2) Lege ein Leeres Set an
	
	3) Für Kante $e$ in $G$
	
	4) Nehme Kante $e$ in Set oder nicht
	
	5) Wenn $e$ ins Set genommen wurde erhöhe $n$ um Eins
	
	6) Wenn $n$ nicht gleich $V/2$ und $e$ nicht letzte Kante in $G$ setzte bei 3) mit nächster Kante fort. 
	
	7)  Wenn $n \neq V/2$ verwerfe.
	
	8)  Wähle eine Permutation des Sets in eine Liste $l$
	
	9) Prüfe Bildet $l$ einen gültigen Pfad mit Anfang = Ende
	
	10) Prüfe ob jeder nicht Endknoten nur einmal auf dem Pfad liegt.
	
	Dann hat M Polynomiele Lauftzeit.
	