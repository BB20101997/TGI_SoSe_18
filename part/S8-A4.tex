\section*{H4}

\subsection*{1)}

Die Aussage in Worten: Es gibt keine Transition zum Startzustand.
$N$ erweitern wir nun um einen neuen Startzustand $q_{0M}$ und entsprechende Transitionen zu $M_1$.

$q_{0_N} \in F_N \Leftrightarrow q_{0_{M_i}} \in F_{M_1}$ 

$F_N  =  F_{M_1} \setminus \{q_{0_{M_i}}\}$

$Q_{M_1} = Q_N \cup \{q_{0_{M_i}}\}$

$\forall q \in Q_{M_1}, a \in \Sigma: \delta_{M_1}(q,a) = 
\begin{cases}
\delta_N(q,a) \textit{ , für }q \neq q_{0_{M_1}} \\
\delta_N(q_{0_N},a) \textit{ , für }q = q_{0_{M_1}}
\end{cases}$ 

\subsection*{2)}

Die Aussage in Worten: Alle Zustände für die eine ausgehende Transition existiert sind nicht akzeptierende Zustände.

Da bei DEA's für jeden Zustand und jedes Eingabe Symbol ein Folge Zustand durch $\delta_{M_2}$ definiert sein muss existiert immer eine  ausgehende Transition, somit darf kein Zustand Endzustand sein d.h. $M_1$ kann nur für $L(N) = \varnothing$  definiert werden als ein Automaten mit einem nicht akzeptierendem Zustand und einem Self-Loop für alle Eingabesymbole.


\subsection*{3)}

Die Aussage in Worten: Wenn zwei Transitionen zum selben Zustand führen dann waren die Eingabesymbole identisch.

Wir erstellen für jedes Eingabesymbol eine Kopie jedes Zustandes.

$Q_{M_3} = \{q_{(q_N,a)}|q_N \in Q_N \wedge a \in \Sigma \}$

Wähle einen der Zustände mit $q_{0_N}$ im Index als Startzustand.

$q_{0_{M_3}} \in \{q(q_{0_N},a) \in Q_{M_3}| a \in \Sigma \}$

$\forall q_{(q_N,a_q)} \in Q_{M_3}, a \in \Sigma: 
\delta_{M_3}((q_N,a_q),a) = q_{(\delta_N(q_N,a),a)} $


$F_{M_3} = \{q_{(q_N,a)} \in Q_{M_3} | q_N \in F_N\}$

\subsection*{4)}

Die Aussage in Worten: Wenn ein Zustand q zwei Transitionen (dürfen auch identisch sein) besitzt sind die Eingabesymbole der Transitionen gleich.

Da für jeden Zustand und jedes Eingabesymbol beim DEA eine Transition definiert sein muss ist für $|\Sigma|>1$ die Aussage immer Falsch und für $|\Sigma| = 1$ immer Wahr.