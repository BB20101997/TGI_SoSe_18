\section*{Aufgabe 3}

\subsection*{a)}

\subsubsection*{1)}

Sei $L_1 = \{\langle M,N \rangle | M \textit{ und } N \textit{sind Truingmachinen und } L(M) \cap L(N) \neq \emptyset\}$

\paragraph*{Behauptung}

$L_1$ ist nicht Turing-entscheidbar

\paragraph*{Beweis}~

Sei $\forall w \in \Sigma: K_w$ die Turing Maschine welche nur für die Eingabe w akzeptiert.

Dann gilt mit f: $HALT_{TM}  \to L_1,  \langle M, w \rangle \mapsto \langle M^\prime, K_w \rangle$ wobei $M^\prime$ aus $M$ hervorgeht indem alle verwerfenden Zustände in akzeptierende abgeändert werden.:

$w \in HALT_{TM} \Leftrightarrow f(w) \in L_1$

$\Rightarrow HALT_{TM} \leq L_1$

Da für $HALT_{TM}$ bekannt ist das es nicht Turing-entscheidbar ist, kann damit auch $L_1$ nicht Turing-entscheidbar sein.

\subsubsection*{2)}

Sei $L_2 = \{ \langle M \rangle | M \textit{ ist Turingmaschine und }\varphi_M(0) = \langle M \rangle \}$

\paragraph*{Behauptung}

$L_2$ ist nicht Turing-entscheidbar.

\paragraph*{Beweis}~

Sei $f: HALT_{TM} \to L_2, \langle M, w \rangle \mapsto \langle M^\prime \rangle$ wobei $M^\prime$
wie folgt arbeitet:

(1) Lösche die Eingabe vom Band

(2) Schreibe $w$ aufs Band

(3) Simuliere $M$ 

(4) Lösche das Band

(5) Schreibe $\langle M^\prime \rangle$ aufs Band

Dann gilt $w \in HALT_{TM} \Leftrightarrow f(w) \in L_2$

$\Rightarrow HALT_{TM} \leq L_2$ 

Da für $HALT_{TM}$ bekannt ist das es nicht Turing-entscheidbar ist, kann damit auch $L_2$ nicht Turing-entscheidbar sein.  

\subsection*{b)}

\subsubsection*{3)}
$L_3=\{\langle M \rangle | M$ ist TM und $\exists w \in \{0,1\}^* \varphi_M(w)=w\}$ ist trivialerweise äquivalent zu $F = \{ \varphi_M | \exists w \in \{0,1\}^* \varphi_M(w)=w \}$
und $L(F)=\{ \langle M \rangle | M$ ist $TM$ und $\varphi_M \in F \}$

$F$ Beinhaltet nur berechenbare Funktionen daher ist nur noch zu Zeigen:

$L(F) \neq \mathbb{F} \wedge L(F) \neq \emptyset $

\paragraph*{Behauptung 1)}
$L(F) \neq \mathbb{F}$

\paragraph*{Beweis 1)}~

Sei M TM die an die erste stelle des Wortes das inverse Zeichen schreibt, oder falls das Wort leer ist 1 an die erste stelle des Bandes, dann treten folgende Fälle ein 
\paragraph*{$w=\varepsilon$}
Dann gilt $\varphi_M(w)=1\neq w$ 
\paragraph*{$w \in \{0,1\}^* \backslash \{ \varepsilon \} $}
Dann existiert $w^{\prime} \in  \{0,1\}^*$ sodass $w=0w^{\prime}$ oder $w=1w^{\prime}$, dann gilt im ersten Fall: $\varphi_M(w)=1w^{\prime} \neq 0w^{\prime}=w$. Und im zweiten Fall: $\varphi_M(w)=0w^{\prime} \neq 1w^{\prime}=w$

Somit gilt für jedes Wort w in $\{0,1\}^*$ dass $\varphi_M(w) \neq w$ somit ist  $\langle M \rangle \notin L(F)$ was bedeutet $L(F) \neq \mathbb{F}$.

\paragraph*{Behauptung 2)}
$L(F) \neq \emptyset$

\paragraph*{Beweis 2)}

Sei M TM die 1 Zeichen einliest, hält und es ausgibt, dann gilt $\langle M \rangle \in L(F)$ da für $w=1$ gilt $\varphi_M(w)=1=w$

\paragraph*{Folgerung}

Laut dem Satz von Rice ist somit $L(F)$ nicht turing erkennbar und somit ist $L_3$ nicht turing erkennbar.


\subsubsection*{4)}
$L_4=\{\langle M \rangle | M$ ist TM und $\forall n \in \mathbb{N} \varphi_M(bin(n))=bin(7*n+7)\}$ ist trivialerweise äquivalent zu $F = \{ \varphi_M | \forall n \in \mathbb{N} \varphi_M(bin(n))=bin(7*n+7) \}$
und $L(F)=\{ \langle M \rangle | M$ ist $TM$ und $\varphi_M \in F \}$

$F$ Beinhaltet nur berechenbare Funktionen daher ist nur noch zu Zeigen:

$L(F) \neq \mathbb{F} \wedge L(F) \neq \emptyset $

\paragraph*{Behauptung 1)}
$L(F) \neq \mathbb{F}$

\paragraph*{Beweis 1)}
Sei $M$ TM die jedes Wort aus $\{0,1\}^*$ unverändert ausgibt, dann gilt für $w=1=bin(1)$ das folgende $\varphi_M(w)=bin(14)=1110 \neq 1$ somit gilt $\langle M \rangle \notin L(F)$ was bedeutet $L(F) \neq \mathbb{F}$.

\paragraph*{Behauptung 2)}
$L(F) \neq \emptyset$
\paragraph*{Beweis 2)}
Sei $M$ 8-Band TM mit folgenden Eigenschaften:
\begin{enumerate}
\item Kopiere die Eingabe von Band 1 auf die Bänder 2 bis 7
\item Speichere $111=bin(7)$ auf das 8. Band
\item Addiere die Bänder 2 bis 8 auf das 1. Band 
\item gib nur das erste Band aus und halte.
\end{enumerate}
Dann gilt $\langle M \rangle \in L(F) $
\paragraph*{Beweis $\langle M \rangle \in L(F) $}
Wähle $n \in \mathbb{N}$ dann gilt $\varphi_M(bin(n))=bin(n)+bin(n)+bin(n)+bin(n)+bin(n)+bin(n)+bin(n)+bin(7)=bin(n+n+n+n+n+n+n+7)=bin(7*n+7)$ daher folgt $\langle M \rangle \in L(F) $.

Daher Folgt $L(F) \neq \emptyset$ und damit gilt laut dem Satz von Rice, dass $L(F)$ nicht turing erkennbar und somit ist $L_4$ nicht turing erkennbar.

\subsection*{c)}

\subsubsection*{5)}
Sei $M^+$ TM die jedes Wort aus $\{0,1\}^{\prime}$ unverändert Ausgibt, sonst verwerfe, dann gilt $\forall w \in \{0,1\}^{\prime}$ $\varphi_M(w)=w$ Sei $M$ TM die $M^+$ simuliert und Akzeptiert wenn $\varphi_M(w)=w$ 

Da $M^+$ terminiert und $M$ $M^+$ simuliert und einmal vergleicht terminiert M und akzeptiert wenn $w \in L_5$ und verwirft wenn $w \notin L_5$

\subsubsection*{6)}
Sei $M$ 3-Band TM wie folgt:
\begin{enumerate}
\item Speichere 1 auf das 2. Band.
\item Speichere 2 auf das 3. Band.
\item Wenn Eingabe $w$ mit $|w|<1$ gilt akzeptiere.
\item Wenn Eingabe $w$ mit $|w|=1$ und $w=0$ gilt akzeptiere.
\item Wenn Eingabe $w$ mit $|w|=2$ und $w=00$ gilt akzeptiere.
\item Wenn gilt dass das höherwertige Band von 2 oder 3 einen höheren oder gleichen wert hat wie das w lang ist  akzeptiere.
\item überprüfe ob der Wert von w an der Stelle vom Wert des höherwertigen Bandes von 2. 3. gleich null ist, wenn nicht verwerfe. 
\item Addiere das höherwertige Band auf das niedrigwertigere und gehe zu 6.
\end{enumerate}
\paragraph*{Beweis}
Für $w \leq 2$ trivial.


Wenn $w \in L_6$ gilt dass für den wert an 1,2 bereits überprüft wurde dann ist das höherwertige band der darrauffolgenden Iterationen ebenfalls eine Fibonaccizahl und es wird überprüft ob w an der Stelle des Höherwertigen Bandes 0 ist, da es aufhört wenn das höherwertige band größer oder gleich ist und das niedrigwertigere Band überprüft wurde gilt dass jede Fibonaccizahl f wo gilt $f < |w|$ überprüft wurde und da $w \in L_6$ akzeptiert L,
Wenn $w \notin L_6$ gilt dass eine Fibonaccizahl f existiert mit $f<|w|$ und $w_f=1$ da M jede fibonaccizahl überprüft verwirft er wenn das Höherwertige band den Wert f erreicht hat.

Da M akzeptiert wenn $w \in L_6$ und verwirft wenn $w \notin L_6$ ist $L_6$ turing-erkennbar.

