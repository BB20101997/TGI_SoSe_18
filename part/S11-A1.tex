\section*{Aufgabe 1}

Sei $G = (N, \Sigma, P, S)$ eine kontextfreie Grammatik, welche eine Sprache $L$ generiert. Wir definieren $G^{\prime} =
(N, \Sigma, P \cup \{ S \to SS \} , S)$. 

\subsection*{Behauptung}

Dann gilt $ L(G^{\prime}) = L^*$ im allgemeinen nicht.

\subsection*{Beweis}

Sei $G = (\{S\},\{a\},\{S \to a\},S)$.

Dann gilt
	 $L = \{a\}$ und
	 $G^{\prime} = (\{S\},\{a\},\{S\to a , S \to SS\},S)$.
	 
	 Dann gilt  $\varepsilon \notin L(G^\prime)$ offentsichtlich, da $G^\prime$ keine Regel enthält welche eine Nicht-Terminal durch $\varepsilon$ ersetzt.
	 
	 Damit aber, da $\varepsilon \in L^*$ gilt, gilt $L(G)\neq L(G^\prime)$.
	 
	 
	 