\section*{Aufgabe 1}
%%TODO 1c)

Seien $L_1, L_2 \subseteq \{0,1\}^*$ Sprachen.

\subsection*{a)}

Sei $L_1 \leq L_2$.

\subsubsection*{Behauptung}

Es gilt $\overline{L_1} \leq \overline{L_2}$

\subsubsection*{Beweis}

Da $L_1 \leq L_2$ existiert f so dass gilt:

$w \in L_1 \Leftrightarrow f(w) \in L_2$

$\Rightarrow w \notin L_1 \Leftrightarrow f(w) \notin L_2$

$\Rightarrow w \in \overline{L_1} \Leftrightarrow f(w) \in \overline{L_2}$ nach Definition von $\overline{\cdot}$

Das bedeutet $\overline{L_1} \leq \overline{L_2}$.


\subsection*{b)}

Sei $L_1$ Turing-erkennbar und $L_1 \leq \overline{L_1}$ 

\subsubsection*{Behauptung}

$L_1$ ist Turing-entscheidbar


\subsubsection*{Beweis}

Da $L_1 \leq \overline{L_1}$ existiert f so dass gilt:

$w \in L_1 \Leftrightarrow f(w) \in \overline{L_1}$

$\Rightarrow  w \notin L_1 \Leftrightarrow f(w) \notin \overline{L_1}$

$\Rightarrow w \in \overline{L_1} \Leftrightarrow f(w) \in L_1$

Damit ist $\overline{L_1} \leq L_1$ und $\overline{L_1}$ Turing-erkennbar, damit ist die Sprache $L_1$ und ihr Komplement Turing-erkennbar damit ist sie auch Turing-entscheidbar.


\subsection*{c)}

\subsubsection*{Behauptung}

%% Für jedes nicht Turing-entscheidbare Problem $L_1$, 
%% existiert eine nicht Turing-entscheidbares Problem mit $L_2 < L_1$

Es gibt kein nicht Turing-entscheidbares Problem $L_1$, sodass für alle nicht Turing-entscheidbaren
Probleme $L_2$ die Eigenschaft $L_1 \leq L_2$ gilt.

\subsubsection*{Beweis}

\paragraph*{Annahme:} Angenommen die Behauptung gilt nicht, also es gibt ein nicht Turing-entscheidbares Problem $L_1$, sodass für alle nicht Turing-entscheidbaren
Probleme $L_2$ die Eigenschaft $L_1 \leq L_2$ gilt.

\paragraph*{Fall 1:}
$L_1$ ist Turing-erkennbar.

Würde dann $L_1 \leq L_2$ für alle nicht Turing-entscheidbaren Probleme $L_2$ gelten, so auch für $L_2 = \overline{L_1}$, nach 1b) wäre $L_1$ dann aber Turing-entscheidbar, was ein Widerspruch zur Definition von $L_1$ wäre.

\paragraph*{Fall 2:}
$L_1$ ist nicht Turing-erkennbar.

Würde dann $L_1 \leq L_2$ für alle nicht Turing-entscheidbaren Probleme $L_2$ gelten.
Also auch für $L_2$ Turing-erkennbar und nicht Turing-entscheidbar, damit muss aber auch $L_1$ Turing-erkennbar sein. Wir sind also in Fall 1.

\paragraph*{}

Alle Fälle führen zu einem Widerspruch, also ist die Annahme falsch und die Behauptung stimmt.



