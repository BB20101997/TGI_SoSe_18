\section*{Aufgabe 3}

\subsection*{a)}
 Eine erfüllende Belegung von $(a_1 \vee a_2 \vee \neg a_3)\wedge (\neg a_1 \vee  a_2  \vee a_3)$ was Element von $3SAT$ ist, ist eine erfüllende Belegung von:
 \begin{list}{}{}
 \item $0 \leq 1*a_1+1*a_2-1*a_3$
 \item $0 \leq -1*a_1+1*a_2+1*a_3$
 \end{list} ist was Element von $LP_{\{0,1\}} $ ist. \\
 
Da Beide Klausen zu 1 ausgewertet werden müssen betrachte Die erste Klausel mit der ersten Ungleichung:
Die Klausel wertet nur zu eins aus wenn $a_1$ oder $a_2$ zu 1 auswerten oder $a_3$ zu 0. Mit anderen Worten es ist nicht 1 wenn $a_1=a_2=1$ und $a_3=1$. Die Ungleichung kann nur nicht gelten wenn $a_3=1$ da es der einzige negative Factor ist dafür müssen $a_1$ und $a_2$ aber 0 Sein damit die rechte Seite negativ wird, daraus folgt die Ungleichung ist nur ungültig wenn $a_1=a_2=1$ und $a_3=1$. 
Gleiche Begründung für die 2. Klausel und Ungleichung. Da beide Klausel zu 1 Auswerten müssen und beide Ungleichungen erfüllt werden müssen folgt daraus dass eine Erfüllende Belegung von diesem $3SAT$ Problem zu einer erfüllenden Belegung dieses $LP_{\{0,1\}}$ Problems wird.

\subsection*{b)}
Ich nehme an dass Zahlen immer eine Codierung (Binär/Decimal) haben:
\begin{enumerate}
\item Für jede Ungleichung speichere die Codierung der Zahl links vom $\leq$ dann Speichere das $\leq$ Zeichen.
\item für jede Zahl der Form $m_{i,j}$ Speichere ihren Start mit dem Zeichen $m$ dann die Codierung der Zahl mit Vorzeichen.
\item Für jede Variable Speichere das Zeichen $X$ und die Codierung ihres Indexes wenn noch eine Variable folgt führe 2. aus
\item ist die Letzte Variable der Ungleichung codiert markiere das ende der Ungleichung mit \# wenn noch eine weitere Ungleichung existiert gehe zu 1.

\end{enumerate}

Dies ist in Polinominalzeit zu erzeugen da:
\begin{list}{-}{}
\item Der 1. schritt Konstant viel zeit braucht.
\item Der 2. Schritt maximal k mal pro Ungleichung ausgeführt wird und die Codierung maximal polinominell ist 
\item Der 3. Schritt maximal k mal pro Ungleichung ausgeführt wird und die Codierung maximal polinominell ist 
\item Der 4. Schritt Konstant viel zeit verbraucht und die Schritte 1-3 maximal $l$ mal ausführen lässt.

\end{list}

 Daraus folgt diese Codierung in $O(l*k)$ ausgeführt werden kann, also polinomiell.
 
\subsection*{c)} 
Zu zeigen $LP_{\{0,1\}} \in NP$ und $3SAT \leq_p LP_{\{0,1\}}$

Zeige $LP_{\{0,1\}} \in NP$ über polinominellen Verifikator über Eingabe $<<U>,e>$:
\begin{enumerate}
\item Setze i=1
\item Nutze Funktion e auf $U_i$ an (Polinomiell da e polinomiell ($O(k)$))
\item Prüfe ob Ungleichung erfüllt ist wenn nicht verwerfe (Polinomiell da nur Addition und Multiplikation)
\item Wenn $i+1 \leq l$ setze $i=i+1$ und gehe zu 2. , wenn nicht akzeptiere (Polinominell da nur $1$ vergleich und maximal l Widerholungen.) 
\end{enumerate}
Dieser Verifikator prüft ob $e$ eine Lösung für $<U>$ ist in $O(l*k)$ Zeit.\\

Sei $((X_1,\ldots ,X_k),(K_1, \ldots , K_l)) \in 3SAT$ mit $(X_1,\ldots ,X_k)$ als $k$ Variablen und $(K_1, \ldots , K_l)$ als $l$ Klauseln. Dann bestehen die Klauseln aus 3 Literalen die negiert seien können, sei $K_i$ mit $i \in \{ 1, \ldots , l\}$ Klausel der Form $(L_j \vee L_o \vee L_p)$ mit $j,o,p \in \{ 1, \ldots , k\}$ da die Reihenfolge von Literalen egal ist kann O.b.d.A angenommen werden dass $K_i$ 4 Zustände haben kann:
\begin{list}{}{}
\item $(X_j \vee X_o \vee X_p)$
\item $( \neg X_j \vee X_o \vee X_p)$
\item $(\neg X_j \vee \neg X_o \vee X_p)$
\item $( \neg X_j \vee \neg X_o \vee \neg X_p)$
\end{list}
Bilde ein neue Ungleichung $u_i$ mit $n_i=1-($Anzahl $\neg$ Literale$)$ und $m_{i,j}=1$ wenn Literal nicht negiert ist und $m_{i,j}=-1$ wenn doch, für $o,p$ dasselbe.
Dann gilt:

\begin{list}{}{}
\item $(X_j \vee X_o \vee X_p) \Rightarrow 1 \leq 1*X_j+1* X_o + 1*X_p$
\item $( \neg X_j \vee X_o \vee X_p) \Rightarrow  0 \leq -1*X_j+1* X_o + 1*X_p$
\item $(\neg X_j \vee \neg X_o \vee X_p) \Rightarrow -1 \leq -1*X_j-1* X_o + 1*X_p$
\item $( \neg X_j \vee \neg X_o \vee \neg X_p) \Rightarrow -2 \leq -1*X_j-1* X_o - 1*X_p$
\end{list}
Dann gelten die Ungleichungen nur dann nicht wenn die Kauseln nicht gelten.
\begin{list}{}{}
\item[$(X_j \vee X_o \vee X_p)$] $(X_j=0, X_o=0, X_p=0) \Rightarrow 1 \leq 1*0+1* 0 + 1*0=0$ Und $1 \nleq 0$
\item[$( \neg X_j \vee X_o \vee X_p)$] $( X_j=1 , X_o=0, X_p=0) \Rightarrow  0 \leq -1*1+1* 0 + 1*0=-1$ Und $0 \nleq -1$
\item[$(\neg X_j \vee \neg X_o \vee X_p)$] $(X_j=1, X_o=1, X_p=0) \Rightarrow -1 \leq -1*1-1* 1 + 1*0=-2$ Und $-1 \nleq -2$
\item[$( \neg X_j \vee \neg X_o \vee \neg X_p)$] $( X_j=1, X_o=1,X_p=1) \Rightarrow -2 \leq -1*1-1* 1 - 1*1=-3$ und $-2 \nleq -3$
\end{list}
Die Rechten Seiten der Ungleichungen sind bei diesen Eingaben um 1 kleiner als die linken Seiten und alle Änderungen der Variablen erhöhen den Wert des Terms um 1 das heißt die Ungleichungen gelten für gültige Belegungen der Klauseln und und Gelten nicht für ungültige Belegungen. Diese Transformation einer Klausel funktioniert in Konstanter Zeit da Maximal 3 Literale pro Klausel Vorkommen. Nachdem dieser Vorgang für Alle Klauseln durchgeführt wurde liegt diese Transformation in $O(l)$ da es $l$ Klauseln gibt.

Da für eine gültige Belegung von $3SAT$ alle Klauseln gelten müssen und zu jeder Klausel eine Ungleichung existiert die genau dann gilt wenn die Klausel zu 1 Auswertet liegt diese Transformation in $LP_{\{ 0,1\} }$. Da diese Transformation in $O(l)$ liegt ist die auch Polinominell. $ \Rightarrow 3SAT \leq_p  LP_{\{ 0,1\} }$