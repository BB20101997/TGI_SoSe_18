\section*{Aufgabe 3}

\subsection*{a)}
 Eine erfüllende Belegung von $(a_1 \vee a_2 \vee \neg a_3)\wedge (\neg a_1 \vee  a_2  \vee a_3)$ was Element von $3SAT$ ist, ist eine erfüllende Belegung von:
 \begin{list}{}{}
 \item $0 \leq 1*a_1+1*a_2-1*a_3$
 \item $0 \leq -1*a_1+1*a_2+1*a_3$
 \end{list} ist was Element von $LP_{\{0,1\}} $ ist. \\
 
Da Beide Klausen zu 1 ausgewertet werden müssen betrachte Die erste Klausel mit der ersten Ungleichung:
Die Klausel wertet nur zu eins aus wenn $a_1$ oder $a_2$ zu 1 auswerten oder $a_3$ zu 0. Mit anderen Worten es ist nicht 1 wenn $a_1=a_2=1$ und $a_3=1$. Die Ungleichung kann nur nicht gelten wenn $a_3=1$ da es der einzige negative Factor ist dafür müssen $a_1$ und $a_2$ aber 0 Sein damit die rechte Seite negativ wird, daraus folgt die Ungleichung ist nur ungültig wenn $a_1=a_2=1$ und $a_3=1$. 
Gleiche Begründung für die 2. Klausel und Ungleichung. Da beide Klausel zu 1 Auswerten müssen und beide Ungleichungen erfüllt werden müssen folgt daraus dass eine Erfüllende Belegung von diesem $3SAT$ Problem zu einer erfüllenden Belegung dieses $LP_{\{0,1\}}$ Problems wird.

\subsection*{b)}
Ich nehme an dass Zahlen immer eine Codierung (Binär/Decimal) haben:
\begin{enumerate}
\item Für jede Ungleichung speichere die Codierung der Zahl links vom $\leq$ dann Speichere das $\leq$ Zeichen.
\item für jede Zahl der Form $m_{i,j}$ Speichere ihren Start mit dem Zeichen $m$ dann die Codierung der Zahl mit Vorzeichen.
\item Für jede Variable Speichere das Zeichen $X$ und die Codierung ihres Indexes wenn noch eine Variable folgt führe 2. aus
\item ist die Letzte Variable der Ungleichung codiert markiere das ende der Ungleichung mit \# wenn noch eine weitere Ungleichung existiert gehe zu 1.

\end{enumerate}

Dies ist in Polinominalzeit zu erzeugen da:
\begin{list}{-}{}
\item Der 1. schritt Konstant viel zeit braucht.
\item Der 2. Schritt maximal k mal pro Ungleichung ausgeführt wird und die Codierung maximal polinominell ist 
\item Der 3. Schritt maximal k mal pro Ungleichung ausgeführt wird und die Codierung maximal polinominell ist 
\item Der 4. Schritt Konstant viel zeit verbraucht und die Schritte 1-3 maximal l mal ausführen lässt.

\end{list}

 Daraus folgt diese Codierung in $O(l*k)$ ausgeführt werden kann, also polinomiell.
 
\subsection*{c)} 
Zu zeigen $LP_{\{0,1\}} \in NP$ und $3SAT \leq_p LP_{\{0,1\}}$

Zeige $LP_{\{0,1\}} \in NP$ über polinominellen Verifikator über Eingabe $<<U>,e>$:
\begin{enumerate}
\item Setze i=1
\item Nutze Funktion e auf $U_i$ an (Polinomiell da e polinomiell ($O(k)$))
\item Prüfe ob Ungleichung erfüllt ist wenn nicht verwerfe (Polinomiell da nur Addition und Multiplikation)
\item Wenn $i+1 \leq l$ setze $i=i+1$ und gehe zu 2. , wenn nicht akzeptiere (Polinominell da nur $1$ vergleich und maximal l Widerholungen.) 
\end{enumerate}
Dieser Verifikator prüft ob $e$ eine Lösung für $<U>$ ist in $O(l*k)$ Zeit 

Sei $((X_1,\ldots ,X_k),(K_1, \ldots , K_l)) \in 3SAT$ mit $(X_1,\ldots ,X_k)$ als $k$ Variablen und $(K_1, \ldots , K_l$ als $l$ Klauseln. 