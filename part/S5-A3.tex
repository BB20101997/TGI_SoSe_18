\section*{H3}

\subsection*{a)}
	
Sei $\Sigma$ Alphabet mit \textvisiblespace$ \notin \Sigma$ ,so dass $L = \Sigma^{*}$ unendliche Sprache.

Dann akzeptiert die Turing Maschine 
$T = (\{q_{0},q_{1}\},\Sigma,\Sigma \cup \{$\textvisiblespace$\},\delta,q_{0},\{q_{1}\})$ 

mit $\delta(q_{0},r) = \{(q_{1},r,L)\}$ für alle $r \in \Sigma \cup \{$\textvisiblespace$\}$ die Sprache L.

Also existiert eine Turing Maschine T

welche eine unendliche Sprache akzeptiert ohne 

das der Kopf sich um mehr als einen Schritt von der Startposition entfernt.

\subsection*{b)}

Im folgendem wird die Aussage 
\begin{quote}
	Zu jeder Turing Maschine $M_{1}$ existiert eine Turing Maschine $M_{2}$
	mit nur einem Zustand und $L(M_{1}) = L(M_{2})$.
\end{quote} widerlegt.

Sei $M_{1}$ eine Turing Maschine,

welche das Wort $a$ akzeptiert und das Wort $b$ verwirft.

Dann kann keine Turing Maschine $M_{2}$ existieren,

welche nur einen Zustand hat und für die $L(M_{1})=L(M_{2})$ gilt.

Da $M_{2}$ nur einem Zustand hätte könnte $M_{2}$ entweder alle Wörter akzeptiert

oder alle Wörter verwerfen.

Da für $M_{1}$ sowohl ein Wort akzeptiert als auch verworfen werden muss,

widerlegt dies die Aussage.

\subsection*{c)}

Die hier zu Beweisende Aussage wird durch die Turing Maschine T in a),
welche die Bedingung, dass sich der Kopf nur nach links Bewegt, erfüllt, widerlegt,
da $\varepsilon \in L(T)$, aber $\varepsilon \notin \Sigma^{+}$, und so mit nicht $L(T) \subseteq \Sigma^{+}$

