\section*{Aufgabe 2}

\subsection*{Behauptung}

$FVS \in NPC$

\subsection*{Beweis}

$FVS \in NP$ nach Vorlesung bekannt.

Es bleibt zu zeigen FVS NP-Schwere.

Dazu zeigen wir \textit{Vertex-Cover} $\leq_P FVS$,
da Vertex-Cover NP-Schwere ist damit dann auch FVS NP-Schwere.

Sei $f: \textit{Vertex-Cover} \to FVS, ((V,E),k) \mapsto ((V,E^\prime),k)$ mit
 $E^\prime = \{(a,b) | (a,b) \in E \vee (b,a) \in E  \}$.
 
Zu zeigen bleigt $\forall w \in \textit{Vertex-Cover}$ gilt
$w \in \textit{Vertex-Cover} \Leftrightarrow f(w) \in FVS$

Sei $((V,E),k) =  w$.

Dann ist $((V,E^\prime),k) = f(w)$

\subsubsection*{"$\Rightarrow$"}
Sei $w \in \textit{Vertex-Cover}$.

Dann existiert eine Knoten Menge $t$ welche ein $k$ Vertex-Cover für $(V,E)$ ist.

$\forall (a,b) \in E, \exists v \in t: v \in \{a,b\}$ nach definition von $t$

Sei dann $(a,b) \in E^\prime$, dann ist $(a,b)$ oder $(b,a)$ 
in $E$ nach definition von $E^\prime$ und es gilt $\exists v \in t : v \in \{a,b\}$

Dann ist $t$ auch ein $k$ FVS für $(V,E^\prime)$ da der resultierende Graph durch entfernen der Knoten in $t$ keine Kanten und somit keine Kreise enthält, also $f(w) \in FWS$


\subsubsection*{"$\Leftarrow$"}

Sei $f(w) \in FVS$.

Dann existiert eine Knoten Menge $t$ welche ein $k$ FVS für $(V,E^\prime)$ ist.

Dann gilt $\forall (a,b) \in E^\prime : (b,a) \in E^\prime$ nach definition von $E^\prime$.

D.h. für jede Kannte $(a,b)$ existiert der Kreis $a \rightleftarrows b$.

$\Rightarrow \forall (a,b) \in E^\prime ,\exists v \in t: v \in \{a,b\}$ nach definition von $t$.

$E \subseteq E^\prime$ geht aus der definition von $E^\prime$ direkt hervor.

Also gilt $\forall (a,b) \in E , \exists v \in t : v \in \{a,b\} $

Dann ist $t$ ein $k$ Vektor-Cover für $(V,E)$, also $w \in \textit{Vertex-Cover}$.