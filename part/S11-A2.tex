\section*{Aufgabe 2}

Sei $\Sigma = \{ a, b \}$ ein Alphabet und $L =\{ww^T|w \in \Sigma^+\}\subseteq \Sigma^*$
eine Sprache. Weiter sei die kontextfreie
Grammatik $G = (N, \Sigma, P, S)$ mit $N = \{ S \}$ und $P$ enthält die folgenden Produktionen
$S → aSa | bSb | aa | bb$
gegeben.

\subsection*{Behauptung}

$L = L(G)$

\subsection*{Beweis}

\subsection*{IA}
	
	Sei $w \in \Sigma$,
	
	dann ist $w \in \Sigma^+$ mit $|w| = min(n\in\mathbb{N}| \exists v \in \Sigma^+ : n = |v|)$.
	
	Sei $v = ww^T$.

	Dann ist $v \in L$ und $v = aa$ oder $v = bb$.
	
	Die Wörter in $L(G)$ mit Ableitungen der Länge 1 sind $aa$ und $bb$
	durch die Ableitungsregeln $S \to aa$ bzw. $S \to bb$.
				
\subsection*{IV}
	Sei $v \in L$ und $v \in L(G)$.
	
\subsection*{IS}

	Sei $\eta \in \Sigma$ und $v^\prime = \eta v\eta$.
	
	Nach definition von $L$ und $v$ existiert $w$ mit $v = ww^T$.
	
	Dann ist auch $v^\prime \in L$, da $v^\prime = w^\prime w^{\prime T}$ für $w^\prime = \eta w$.

	Somit $v^\prime \in L$
		
	Nach definition von $G$ und $v$ existiert eine Ableitung A mit $S \Rightarrow^*_A v$.	
	
	Dann ist auch $v' \in L(G)$ da G v' erzeugt mit der Ableitung $S \Rightarrow \eta S\eta \Rightarrow^*_A \eta v\eta = v'$.
	
	Somit ist $\eta v\eta $ auch in $L(G)$.